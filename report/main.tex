\documentclass[oneside, a4paper, onecolumn, 11pt]{article}

% Change this: Customize the title, author, advisor, abstract
\newcommand{\thesistitle}[0]{Emotion Recognition in Conversation through Emotion Flow}
\newcommand{\authorname}[0]{Bruno Iorio}

\newcommand{\supervisor}[0]{Gaël Guibon}
\newcommand{\supervisorinstitution}[0]{LIPN - Université Sorbonne Paris Nord}

\newcommand{\abstracttext}[0]{%
  Emotion Recognition in Conversation (ERC) is a very important domain, which has been gaining more attention in recent years, 
  especially within NLP. In the scope of Emotion Recognition, identifying emotions in dialogues plays an essential role. This is because 
  most of the emotional text data collection happens in the context of a conversation between two or more parties (e.g. customer service survey).
  In this paper we discuss the limitation of previous approches to the ERC task, while also evaluating an original approach
  to the same problem using Causal learning, which we identify as the Emotion Flow and attention mechanisms. (initial version)
}

\usepackage[
  left=2cm,top=2.0cm,bottom=2.0cm,right=2cm,
  headheight=17pt, % as per the warning by fancyhdr
  includehead,includefoot,
  heightrounded, % to avoid spurious underfull messages
]{geometry}


\usepackage[T1]{fontenc}
\usepackage{amstext}
\usepackage{amsmath}
\usepackage{amssymb}
\usepackage{url}
\usepackage{graphicx}
\usepackage{wrapfig}
\usepackage{enumerate}
\usepackage{paralist}
\usepackage{xspace}
\usepackage{color}
\usepackage{times}
\usepackage[colorlinks,linkcolor=blue]{hyperref}
\usepackage[colorinlistoftodos,prependcaption,textsize=normal]{todonotes}
\usepackage{pdfpages}
\usepackage{fancyhdr} %% For changing headers and footers

\usepackage{titling}
\usepackage[nottoc,numbib]{tocbibind}

%% \predate{}
%% \postdate{}
%% \date{}
%% \author{\authorname}


\begin{document}

%\title{\thesistitle}

%\maketitle

% Max 10 lines.
%\noindent \paragraph*{Abstract}
%\abstract

\hspace{0pt}
\vfill

\begin{center}

\includegraphics[width=0.3\textwidth]{logo-EP-vertical}

\vspace*{2em}
%
{\large
\textbf{\'Ecole Polytechnique}

\vspace*{1em}
\textit{BACHELOR THESIS IN COMPUTER SCIENCE}


\vspace*{3em}
{\Huge \textbf{\thesistitle}}
\vspace*{3em}



\textit{Author:}

\vspace*{1em}
\authorname{}, \'Ecole Polytechnique

\vspace*{2em}
%
{\textit{Advisor:}}

\vspace*{1em}
\supervisor{}, \supervisorinstitution{}
}

\vspace*{2em}
\textit{Academic year 2024/2025}

\end{center}

\vfill
\hspace{0pt}

\newpage

\vfill
\noindent\textbf{Abstract}\\[1em]
%
\fbox{
\parbox{\textwidth}{
\abstracttext{}
}
}
\vfill


\newpage

% Setting up the header
\pagestyle{fancy}
%\renewcommand{\headrulewidth}{0pt} % Remove line at top
%\renewcommand{\headrulewidth}{0.4pt}% Default \headrulewidth is 0.4pt
\lhead{\authorname}
%\chead{\acronym}
\rhead{\thesistitle}



\newpage
\tableofcontents
\newpage

%\pagenumbering{arabic}

\section{Introduction}  % I am thinking of adding overview on the topic + Related work

Emotions can be defined as one's psicological state, which can be caused by internal or external factors of an individual.
While studying internal factors for the emotions, such as mental state and background, can be very difficult, external factors
tend to be much more viable for studying, particularly Within the field of Natural Language Processing (NLP). The textual context 
provides very significant information that can directly influence the emotion present in a text. In a dialogue, for example, emotions 
in dialogues will depend heavely on the way that the dialogue progress.

(INSERT DIAGRAM EXEMPLIFYING THE ABOVE)

$$\vdots$$

$$\vdots$$

Emotion Recognition in Conversation (ERC) is a growing field within NLP which tackles the task
of identifying emotions in conversation. Its increasing importance is partially explained by the need of companies to process
and evaluate the level of customer satisfaction, which allows them to offer better services, and to be more competitive in the
market. In fact, the collection of this kind of data through a conversation framework provides a type of information that kind
be explored in very interesting ways (e.g. a slight change in the tone of speach, which could evidenciate frustration). \textbf{(MAYBE BE MORE SPECIFIC
WHY THE CONVERSATION FRAMEWORK IS IMPORTANT)}

Even from a human perspective, identifying emotions can be very difficult, especially when we cannot rely on other factors such
as voice tone, gestures, facial expressions etc. Also, even the classification of emotions can be considered ambiguous sometimes.
For instance, joyfulness, happiness, and love are generally considered similar, and sometimes only being distinguished in terms of
intensity. This disambiguouity is what imposes the challenges in ERC, motivating many different approaches for this task. 

Previous researches have extensively studied how different model architectures affect the performance of the models. 
DialogueRNN \cite{majumder2019dialoguernnattentivernnemotion} uses Gated Recurrent Networks (GRN) to infer a representation for 
each emotion, while applying an attention mechanism to predict an emotion. ERC-DP \cite{wang2021t2vladgloballocalsequencealignment} 


$$\vdots$$

$$\vdots$$

Among the factors that can influence emotion detection, we particularly study the Emotion Flow in the conversation, which can be
denoted as the graph describing the evolution, or change, in emotion throughout the conversation. However, the use of the Emotion 
Flow is limited to an utterance level. Which, while being a very logical way of studying emotions in conversation, has left some 
space for how a word level approach would perform. This is, instead of predicting an emotion for each utterance, a word level approach 
would predict an emotion for word. Indeed, to the extend of our research, there isn't any work evaluating the former approach in the 
task of ERC, leaving us with some room to study it.

In this Project, we aim at evaluating how a word level approach can be combined with the Emotion Flow in the ERC task. This comes with 
the challenge of determining ways to fuse the information given by the Emotion Flow and the flow of words. 

$$\vdots$$

$$\vdots$$

(ADD PICTURE DEPICTING EMOTION FLOW)

\section{Related Work} 
Probably, the most challenging problem in this project is to find efficient ways to fuse the textual and emotional information. 
In \cite{LIU2025126924}, it is applied a window transformer - using narrow 2D window masks - to catch short-term inter-utterance relations, 
used in the task of Causal Emotion Entailment(CEE), e.g. determining which parts of the text are causing each emotion. This was 
particularly inspired from MPEG \cite{10252019}, where the emotional information is embeddeded and later on fused with the textual 
information using an attention layer. This is useful because the CEE is closely related to ERC , and actually its use can actually 
improve the effeciency of ERC models \cite{LIU2025126924}. In 


\section{Discussion (name subject to change)} % So we can discuss the limitiation of previous approaches 

\section{Datasets used}

\section{Our Approach / Methodology (Absolutely going to change this title)}

\section{Limitation of our approach}

\newpage
\nocite{*}
\bibliographystyle{plain}
\bibliography{main}

\newpage
\appendix

\section{Appendix}
\label{sec:appendix}

\end{document}
;
