\documentclass[oneside, a4paper, onecolumn, 11pt]{article}

% Change this: Customize the title, author, advisor, abstract
\newcommand{\thesistitle}[0]{Emotion Recognition in Conversation through Emotion Flow}
\newcommand{\authorname}[0]{Bruno Iorio}

\newcommand{\supervisor}[0]{Gaël Guibon}
\newcommand{\supervisorinstitution}[0]{LIPN - Université Sorbonne Paris Nord}

\newcommand{\abstracttext}[0]{%
  Emotion Recognition in Conversation (ERC) is a very important domain, which has been gaining more attention in recent years, 
  especially within NLP. In the scope of Emotion Recognition, identifying emotions in dialogues plays an essential role. This is because 
  most of the emotional text data collection happens in the context of a conversation between two or more parties (e.g. customer service survey).
  In this paper we discuss the limitation of previous approches to the ERC task, while also evaluating an original approach
  to the same problem using Causal learning, which we identify as the Emotion Flow and attention mechanisms. (initial version)
}

\usepackage[
  left=2cm,top=2.0cm,bottom=2.0cm,right=2cm,
  headheight=17pt, % as per the warning by fancyhdr
  includehead,includefoot,
  heightrounded, % to avoid spurious underfull messages
]{geometry}


\usepackage[T1]{fontenc}
\usepackage{amstext}
\usepackage{amsmath}
\usepackage{amssymb}
\usepackage{url}
\usepackage{graphicx}
\usepackage{wrapfig}
\usepackage{enumerate}
\usepackage{paralist}
\usepackage{xspace}
\usepackage{color}
\usepackage{times}
\usepackage[colorlinks,linkcolor=blue]{hyperref}
\usepackage[colorinlistoftodos,prependcaption,textsize=normal]{todonotes}
\usepackage{pdfpages}
\usepackage{fancyhdr} %% For changing headers and footers

\usepackage{titling}
\usepackage[nottoc,numbib]{tocbibind}

%% \predate{}
%% \postdate{}
%% \date{}
%% \author{\authorname}


\begin{document}

%\title{\thesistitle}

%\maketitle

% Max 10 lines.
%\noindent \paragraph*{Abstract}
%\abstract

\hspace{0pt}
\vfill

\begin{center}

\includegraphics[width=0.3\textwidth]{logo-EP-vertical}

\vspace*{2em}
%
{\large
\textbf{\'Ecole Polytechnique}

\vspace*{1em}
\textit{BACHELOR THESIS IN COMPUTER SCIENCE}


\vspace*{3em}
{\Huge \textbf{\thesistitle}}
\vspace*{3em}



\textit{Author:}

\vspace*{1em}
\authorname{}, \'Ecole Polytechnique

\vspace*{2em}
%
{\textit{Advisor:}}

\vspace*{1em}
\supervisor{}, \supervisorinstitution{}
}

\vspace*{2em}
\textit{Academic year 2023/2024}

\end{center}

\vfill
\hspace{0pt}

\newpage

\vfill
\noindent\textbf{Abstract}\\[1em]
%
\fbox{
\parbox{\textwidth}{
\abstracttext{}
}
}
\vfill


\newpage

% Setting up the header
\pagestyle{fancy}
%\renewcommand{\headrulewidth}{0pt} % Remove line at top
%\renewcommand{\headrulewidth}{0.4pt}% Default \headrulewidth is 0.4pt
\lhead{\authorname}
%\chead{\acronym}
\rhead{\thesistitle}



\newpage
\tableofcontents
\newpage

%\pagenumbering{arabic}

\section{Introduction}  % I am thinking of adding overview on the topic + Related work

Emotions can be defined as one's psicological state. Sometimes, it can be very subtle, volatile,
and even ambiguous depending on the context. For instance, joyfulness, happiness, and love are generally
considered similar, sometimes only begin distinguished by intensity. So, the task of identifying these emotions
are not always trivial even from a human perspective, so this is why Emotion Recognition has become so important
recently. Within this area, Emotion recognition in text plays a significant role, since it can be leveraged in 
many different contexts, e.g. customer's service, catalogue algorithms, etc (add more examples).

In this context, studying Emotion Recognition in Conversation (ERC) is very relevant, because the flow of 
words can generate a flow of emotions, and they work together to convey meaning to a conversation. This is a very challenging 
task, because exploiting contextual information has its limitations. And previous works haven't explored properly how the 
flow of emotions in a conversation (e.g. the graph describing the emotional changes throughout a conversation) relate with 
contextual information. 

(ADD PICTURE DEPICTING EMOTION FLOW)

\noindent 

\section{Related Work} 

\section{Discussion (name subject to change)} % So we can discuss the limitiation of previous approaches 

\section{Datasets used}


\section{Our Approach / Methodology (Absolutely going to change this title)}

\newpage
\nocite{*}
\bibliographystyle{plain}
\bibliography{main}

\newpage
\appendix

\section{Appendix}
\label{sec:appendix}

\end{document}

