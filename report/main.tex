\documentclass[oneside, a4paper, onecolumn, 11pt]{article}

% Change this: Customize the title, author, advisor, abstract
\newcommand{\thesistitle}[0]{Emotion Recognition in Conversation through Emotion Flow}
\newcommand{\authorname}[0]{Bruno Iorio}

\newcommand{\supervisor}[0]{Gaël Guibon}
\newcommand{\supervisorinstitution}[0]{LIPN - Université Sorbonne Paris Nord}

\newcommand{\abstracttext}[0]{%
  Emotion Recognition in Conversation (ERC) is a very important domain, which has been gaining more attention in recent years, 
  especially within NLP. In the scope of Emotion Recognition, identifying emotions in dialogues plays an essential role. This is because 
  most of the emotional text data collection happens in the context of a conversation between two or more parties (e.g. customer service survey).
  In this paper we discuss the limitation of previous approches to the ERC task, while also evaluating an original approach
  to the same problem using Causal learning, which we identify as the Emotion Flow and attention mechanisms. (initial version)
}

\usepackage[
  left=2cm,top=2.0cm,bottom=2.0cm,right=2cm,
  headheight=17pt, % as per the warning by fancyhdr
  includehead,includefoot,
  heightrounded, % to avoid spurious underfull messages
]{geometry}


\usepackage[T1]{fontenc}
\usepackage{amstext}
\usepackage{amsmath}
\usepackage{amssymb}
\usepackage{url}
\usepackage{graphicx}
\usepackage{wrapfig}
\usepackage{enumerate}
\usepackage{paralist}
\usepackage{xspace}
\usepackage{color}
\usepackage{times}
\usepackage[colorlinks,linkcolor=blue]{hyperref}
\usepackage[colorinlistoftodos,prependcaption,textsize=normal]{todonotes}
\usepackage{pdfpages}
\usepackage{fancyhdr} %% For changing headers and footers

\usepackage{titling}
\usepackage[nottoc,numbib]{tocbibind}

%% \predate{}
%% \postdate{}
%% \date{}
%% \author{\authorname}


\begin{document}

%\title{\thesistitle}

%\maketitle

% Max 10 lines.
%\noindent \paragraph*{Abstract}
%\abstract

\hspace{0pt}
\vfill

\begin{center}

\includegraphics[width=0.3\textwidth]{logo-EP-vertical}

\vspace*{2em}
%
{\large
\textbf{\'Ecole Polytechnique}

\vspace*{1em}
\textit{BACHELOR THESIS IN COMPUTER SCIENCE}


\vspace*{3em}
{\Huge \textbf{\thesistitle}}
\vspace*{3em}



\textit{Author:}

\vspace*{1em}
\authorname{}, \'Ecole Polytechnique

\vspace*{2em}
%
{\textit{Advisor:}}

\vspace*{1em}
\supervisor{}, \supervisorinstitution{}
}

\vspace*{2em}
\textit{Academic year 2023/2024}

\end{center}

\vfill
\hspace{0pt}

\newpage

\vfill
\noindent\textbf{Abstract}\\[1em]
%
\fbox{
\parbox{\textwidth}{
\abstracttext{}
}
}
\vfill


\newpage

% Setting up the header
\pagestyle{fancy}
%\renewcommand{\headrulewidth}{0pt} % Remove line at top
%\renewcommand{\headrulewidth}{0.4pt}% Default \headrulewidth is 0.4pt
\lhead{\authorname}
%\chead{\acronym}
\rhead{\thesistitle}



\newpage
\tableofcontents
\newpage

%\pagenumbering{arabic}

\section{Introduction}  % I am thinking of adding overview on the topic + Related work

Emotions can be defined as one's psicological state, which can be caused by internal or external factors of an individual.
Whereas studying internal factors for the emotions of an individual, such as mental state and background, can be difficult, 
understanding external is a very interesting problem, and also tends to be much more viable in the scope of Natural Language 
Processing (NLP). For instance, the textual context in which a dialogue occurs plays a significant role in how each of the 
conversation parties will both absorb the same information, which may derive different reactions, or emotions.

Furthermore, in the context of an interaction wwe can derive the concept of Emotion Flow, which is the graph that describes 
the change in emotions throughout an exchange between two or more parties. It can be particularly useful when studying 

(INSERT DIAGRAM EXEMPLIFYING THE ABOVE)

$$\vdots$$

$$\vdots$$

Emotion Recognition in Conversation (ERC) is a growing field within NLP which tackles the task
of identifying emotions in conversation. Its increasing importance is partially explained by the need of companies to process
and evaluate the level of customer satisfaction, which allows them to offer better services, and to be more competitive in the
market. In fact, the collection of this kind of data through a conversation framework provides a type of information that kind
be explored in very interesting ways (e.g. a slight change in the tone of speach, which could evidenciate frustration). \textbf{(MAYBE BE MORE SPECIFIC
WHY THE CONVERSATION FRAMEWORK IS IMPORTANT)}

Even from a human perspective, identifying emotions can be very difficult, especially when we cannot rely on other factors such
as voice tone, gestures, facial expressions etc. Also, even the classification of emotions can be considered ambiguous sometimes.
For instance, joyfulness, happiness, and love are generally considered similar, and sometimes only being distinguished in terms of
intensity. This disambiguouity is what imposes the challenges in ERC, motivating many different approaches for this task.

Previous researches have extensively studied how different model architectures affect the performance of the models. (ADD EXAMPLES OF 
ARCHITECTURES). 

$$\vdots$$

$$\vdots$$

However, as of today, there is not much reasearch on how the Emotion Flow can affect a model's performance.


$$\vdots$$

$$\vdots$$

(ADD PICTURE DEPICTING EMOTION FLOW)

\section{Related Work} 

\section{Discussion (name subject to change)} % So we can discuss the limitiation of previous approaches 

\section{Datasets used}

\section{Limitation of our approach}
\section{Our Approach / Methodology (Absolutely going to change this title)}

\newpage
\nocite{*}
\bibliographystyle{plain}
\bibliography{main}

\newpage
\appendix

\section{Appendix}
\label{sec:appendix}

\end{document}

